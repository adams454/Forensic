\documentclass[memoire, 12pt]{report}

% === Encodage & langue ===
\usepackage[utf8]{inputenc}
\usepackage[T1]{fontenc}
\usepackage[french]{babel}

% === Mise en page ===
\usepackage[top=1.9cm, bottom=1.5cm, left=1.9cm, right=2.1cm]{geometry}
\usepackage{setspace}
\usepackage{ragged2e}
\usepackage{float}
\usepackage[bottom]{footmisc}
\usepackage[section]{placeins}

% === Mathématiques ===
\usepackage{amsmath, amssymb, amsfonts}

% === Tableaux et mise en forme ===
\usepackage{array, tabularx, longtable, multirow}
\usepackage[table,xcdraw]{xcolor}
\usepackage{caption}
\usepackage{subcaption}

% === Graphiques et dessins ===
\usepackage{graphicx}
\usepackage{tikz}
\usepackage[export]{adjustbox}

% === Références et bibliographie ===
\usepackage{multibib}
\newcites{biblio}{Bibliographie}
\newcites{other}{Autres références}
\usepackage[babel=true]{csquotes}
\usepackage{url}
\usepackage{pdfpages}

% === Algorithmes ===
%\usepackage{algorithm}
%\usepackage{algorithmic}
% Si tu préfères algorithm2e, commente les deux lignes ci-dessus et décommente celle-ci :
\usepackage[ruled,vlined,french,onelanguage]{algorithm2e}

% === Autres utilitaires ===
\usepackage{lmodern}
\usepackage{rotating}
\usepackage{lipsum}
\usepackage{minted}   % pour le code source coloré
\usepackage{listings}
\usepackage[normalem]{ulem} % pour \uline etc.
\useunder{\uline}{\ul}{}

% === Glossaires ===
\usepackage{glossaries}

% === Liens hypertextes ===
\usepackage{hyperref}


% Augmente l'espacement vertical entre les entrées
\setlength{\cftbeforesecskip}{8pt}   % espace avant chaque section
\setlength{\cftbeforesubsecskip}{4pt} % espace avant chaque sous-section
% Met en gras les sections principales dans la table des matières
\renewcommand{\cftsecfont}{\bfseries}
\renewcommand{\cftsecpagefont}{\bfseries}

% Optionnel : augmente l'espacement entre les points de la ligne
\renewcommand{\cftdotsep}{2}


\renewcommand{\thesection}{\Roman{section}} 
% Configuration des styles pour le code Python

\definecolor{codegreen}{rgb}{0,0.6,0}
\definecolor{codegray}{rgb}{0.5,0.5,0.5}
\definecolor{codepurple}{rgb}{0.58,0,0.82}
\definecolor{backcolour}{rgb}{0.95,0.95,0.92}

\lstdefinestyle{python}{
    backgroundcolor=\color{backcolour},   
    commentstyle=\color{codegreen},
    keywordstyle=\color{magenta},
    numberstyle=\tiny\color{codegray},
    stringstyle=\color{codepurple},
    basicstyle=\ttfamily\footnotesize,
    breakatwhitespace=false,         
    breaklines=true,                 
    captionpos=b,                    
    keepspaces=true,                 
    numbers=left,                    
    numbersep=5pt,                  
    showspaces=false,                
    showstringspaces=false,
    showtabs=false,                  
    tabsize=2
}

\lstset{style=python}
%\usepackage{fancyhdr}
\usepackage[Conny]{fncychap}
%Conny
%Bjornstrup
%\pagestyle{Conny}
\usepackage[french]{babel}
%\renewcommand{\footrulewidth}{3pt}
\makeglossaries
\title{Document_De_KALDADAK_ADAMA}
\author{}
\date{MOIS_ICI 2025}

\begin{document}
\begin{titlepage}

	\begin{tikzpicture}[remember picture,overlay,inner sep=0,outer sep=0]
		\draw[orange!90!orange,line width=4pt] ([xshift=-1.5cm,yshift=-2cm]current page.north east) coordinate (A)--([xshift=1.5cm,yshift=-2cm]current page.north west) coordinate(B)--([xshift=1.5cm,yshift=2cm]current page.south west) coordinate (C)--([xshift=-1.5cm,yshift=2cm]current page.south east) coordinate(D)--cycle;
		
		\draw ([yshift=0.5cm,xshift=-0.5cm]A)-- ([yshift=0.5cm,xshift=0.5cm]B)--
		([yshift=-0.5cm,xshift=0.5cm]B) --([yshift=-0.5cm,xshift=-0.5cm]B)--([yshift=0.5cm,xshift=-0.5cm]C)--([yshift=0.5cm,xshift=0.5cm]C)--([yshift=-0.5cm,xshift=0.5cm]C)-- ([yshift=-0.5cm,xshift=-0.5cm]D)--([yshift=0.5cm,xshift=-0.5cm]D)--([yshift=0.5cm,xshift=0.5cm]D)--([yshift=-0.5cm,xshift=0.5cm]A)--([yshift=-0.5cm,xshift=-0.5cm]A)--([yshift=0.5cm,xshift=-0.5cm]A);
		
		
		\draw ([yshift=-0.3cm,xshift=0.3cm]A)-- ([yshift=-0.3cm,xshift=-0.3cm]B)--
		([yshift=0.3cm,xshift=-0.3cm]B) --([yshift=0.3cm,xshift=0.3cm]B)--([yshift=-0.3cm,xshift=0.3cm]C)--([yshift=-0.3cm,xshift=-0.3cm]C)--([yshift=0.3cm,xshift=-0.3cm]C)-- ([yshift=0.3cm,xshift=0.3cm]D)--([yshift=-0.3cm,xshift=0.3cm]D)--([yshift=-0.3cm,xshift=-0.3cm]D)--([yshift=0.3cm,xshift=-0.3cm]A)--([yshift=0.3cm,xshift=0.3cm]A)--([yshift=-0.3cm,xshift=0.3cm]A);

	\end{tikzpicture}
	\begin{center}
		\begin{tabular}{l*{40}{@{\hskip.05mm}c@{\hskip.8mm}} c c}
			\begin{tabular}{c}
				
		\footnotesize{\textbf{R\'EPUBLIQUE DU CAMEROUN}} \\
				
				\scriptsize{\textbf{****************}} \\
				
					\scriptsize{\textbf{Paix - Travail - Patrie}} \\
				
			\scriptsize{\textbf{******************}}\\ 
			\footnotesize{	\textbf{UNIVERSIT\'E DE YAOUND\'E I}}\\
				
			\scriptsize{	\textbf{****************}} \\
				
			\footnotesize{	\textbf{ECOLE NATIONALE SUPERIEURE}} \\
			\footnotesize{	\textbf{POLYTECHNIQUE DE YAOUNDE}} \\
				
			\scriptsize{	\textbf{****************}} \\
		   \scriptsize{	\textbf{D\'EPARTEMENT DE GENIE}}\\
		   \scriptsize{	\textbf{INFORMATIQUE}}\\
				
			\scriptsize{	\textbf{****************}}\\
				
			\end{tabular} &
			\begin{tabular}{c}
				
				\includegraphics[height=4cm, width=2.8cm]{logoUY1-eps-converted-to-1.pdf}
				
			\end{tabular} &
			\begin{tabular}{c}
				
				\footnotesize{\textbf{ REPUBLIC OF CAMEROON}} \\
				
				\footnotesize{\textbf{****************}} \\
				
					\scriptsize{\textbf{Peace - Work - Fatherland}} \\
				
				\scriptsize{\textbf{****************}} \\
				\footnotesize{\textbf{UNIVERSITY OF YAOUNDE I}}\\
				
				\scriptsize{\textbf{****************}} \\
				
				\footnotesize{\textbf{NATIONAL ADVANCED SCHOOL}} \\
				\footnotesize{\textbf{OF ENGINEERING OF YAOUNDE}} \\
				
				\scriptsize{\textbf{****************}} \\
				\scriptsize{\textbf{DEPARTMENT OF COMPUTER}}\\
				\scriptsize{\textbf{ENGINEERING}}\\
				
				\footnotesize{\textbf{****************}}\\
				
			\end{tabular}	
		\end{tabular}
	
		\vspace{0.5cm}
		\begin{tabular}{l*{40}{@{\hskip 3.5cm}c@{\hskip5cm}} p{3.5cm} r}
		\end{tabular}
		
		\noindent\rule{\textwidth}{0.7mm}
		\Large{{\textbf{}}}\\
		\Large{{\textbf{\textit{3 HYPOTHESE DE MEURTRE}}}}
		\noindent\rule{\textwidth}{0.7mm}
	\end{center}
		
	\begin{center}
	\begin{tabular}{c}
		
		\vspace{0.1cm}
		\normalsize
	
	
		\vspace{0.1cm}
		\normalsize\textbf{Option }:\\			
		\textsl{Cybersécurité et Investigation Numérique}
		
	\end{tabular}
	\end{center}
		
	\begin{center}
		\normalsize %\hspace{-2cm}
		\begin{tabular}{c}
			\vspace{0.07cm}
			\hspace{0.02cm} \textbf{\textbf{Rédigé par :}}\\
			\hspace{0.02cm} \textsl{\textbf{}}\\
            \hspace{0.02cm} \textsl{\textbf{KALDADAK ADAMA}, 24P824}\\\\
			
		\end{tabular}
	\end{center}
	
	\begin{center}
	\hspace{0.02cm} \textbf{Sous l'encadrement de:}\\
	\hspace{0.02cm} \textsl{M. Minka THierry}
	\end{center}
	
    
	\vspace{4cm}
	\begin{center}
		\textbf{Année académique 2025 / 2026}
	\end{center}
		
	\vspace{-1.4cm}
	
		
	\vfill%\null
	
\end{titlepage}


\section*{Hypothèse 1:Assassinat commandité avec préméditation (mobile : atteinte aux intérêts d’un groupe puissant)}

Cette hypothèse considère que :
\begin{itemize}
    \item un commanditaire (ou un groupe) a décidé délibérément d’éliminer Martinez Zogo ;
    \item un commando organisé a été envoyé pour exécuter l’opération ;
    \item les actes de torture visaient à envoyer un message d’intimidation ;
    \item la logistique (véhicules, coordination téléphonique, déplacement du corps) indique un plan préparé à l’avance.
\end{itemize}


\section*{Hypothèse 2: une opération d'intimidation qui tourne mal}

Ici, la mort n’était pas le but initial :
\begin{itemize}
    \item le commando avait pour objectif d’intimider, « corriger », ou extorquer des informations ;
    \item les actes de violence étaient destinés à faire pression, non à tuer ;
    \item la mort résulte d’un excès de violence, d’une mauvaise évaluation des risques physiologiques, ou d’un dérapage interne ;
    \item certains acteurs pourraient ne pas avoir été informés d’un projet homicide.
\end{itemize}


\section*{Hypothèse 3: Exécution interne au sein d’un réseau (règlement de comptes interne)}

Cette hypothèse se base sur une autre dynamique criminelle :
\begin{itemize}
    \item Martinez Zogo aurait été en contact avec plusieurs sources internes ;
    \item certaines personnes impliquées pouvaient considérer qu’il détenait des informations compromettantes ;
    \item l’opération ne viserait pas à défendre des intérêts extérieurs, mais à régler un conflit interne à un groupe ou une institution ;
    \item la violence extrême serait utilisée pour empêcher la fuite d’informations et éliminer un risque interne.
\end{itemize}


\vspace{0.5cm}




\end{document}